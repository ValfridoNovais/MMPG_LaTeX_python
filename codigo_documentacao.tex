\documentclass{article}%
\usepackage[T1]{fontenc}%
\usepackage[utf8]{inputenc}%
\usepackage{lmodern}%
\usepackage{textcomp}%
\usepackage{lastpage}%
%
\title{Documentação dos Códigos: Inicial e Melhorado}%
\author{Especialista Python}%
\date{\today}%
%
\begin{document}%
\normalsize%
\maketitle%
\section{Introdução}%
\label{sec:Introduo}%
Esta documentação descreve dois códigos, "CODIGO INICIAL" e "CODIGO MELHORADO", que realizam a detecção de faces usando MediaPipe. Ambos os códigos capturam imagens da webcam, detectam faces e salvam fotos de intrusos. O "CODIGO MELHORADO" introduz melhorias no reconhecimento facial e na gestão de arquivos.

%
\section{Documentação do CODIGO INICIAL}%
\label{sec:DocumentaodoCODIGOINICIAL}%
O "CODIGO INICIAL" é responsável por detectar uma face na tela e salvar fotos de intrusos quando uma face diferente da memorizada é detectada.%
\subsection{1. Importação de Bibliotecas}%
\label{subsec:1.ImportaodeBibliotecas}%
O código começa importando as bibliotecas essenciais: OpenCV para captura de vídeo, MediaPipe para a detecção de malhas faciais, Numpy para cálculos numéricos e outras bibliotecas para manipulação de arquivos e tempo.

%
\subsection{2. Funções de Comparação e Memorização de Faces}%
\label{subsec:2.FunesdeComparaoeMemorizaodeFaces}%
As funções criadas no código incluem:%
\begin{enumerate}%
\item%
`compare\_faces()`: Compara duas faces baseadas nos pontos de referência detectados.%
\item%
`ask\_memorize\_face()`: Abre um popup perguntando se o usuário deseja memorizar uma face detectada.%
\item%
`create\_directory\_structure()`: Cria pastas para armazenar as imagens, organizando por ano, mês e dia.%
\item%
`save\_intruder\_photo()`: Salva a imagem do intruso detectado.%
\item%
`save\_memorized\_photo()`: Salva a imagem de uma face memorizada.%
\end{enumerate}

%
\subsection{3. Captura e Processamento de Vídeo}%
\label{subsec:3.CapturaeProcessamentodeVdeo}%
O código inicializa a webcam com `cv2.VideoCapture(0)` e processa cada frame, detectando faces e desenhando a malha facial com MediaPipe. Se uma face for detectada, o código compara com as faces memorizadas e salva a imagem do intruso, se for uma face nova.

%
\section{Documentação do CODIGO MELHORADO}%
\label{sec:DocumentaodoCODIGOMELHORADO}%
O "CODIGO MELHORADO" aprimora o processo de detecção e gerenciamento de faces. Ele adiciona suporte para múltiplas faces, melhorias na comparação facial e armazenamento de imagens.%
\subsection{1. Alterações Importantes}%
\label{subsec:1.AlteraesImportantes}%
O código foi modificado para detectar múltiplas faces, armazenar várias faces memorizadas e garantir que apenas faces não memorizadas sejam salvas como intrusas.

%
\subsection{2. Detecção de Múltiplas Faces}%
\label{subsec:2.DetecodeMltiplasFaces}%
`max\_num\_faces` foi ajustado para detectar várias faces simultaneamente, aumentando a robustez do sistema.

%
\subsection{3. Gerenciamento de Faces Memorizadas}%
\label{subsec:3.GerenciamentodeFacesMemorizadas}%
Agora, múltiplas faces podem ser memorizadas e o código consegue comparar a face detectada com todas as memorizadas antes de salvar como intruso.

%
\section{Comparação entre o CODIGO INICIAL e CODIGO MELHORADO}%
\label{sec:ComparaoentreoCODIGOINICIALeCODIGOMELHORADO}%
As principais diferenças entre os dois códigos estão no tratamento de múltiplas faces e na otimização do processo de comparação e salvamento de imagens.%
\begin{enumerate}%
\item%
Suporte para múltiplas faces no "CODIGO MELHORADO".%
\item%
Armazenamento eficiente de imagens organizadas por data e hora.%
\item%
Maior precisão na comparação de faces.%
\end{enumerate}

%
\end{document}